\documentclass[a4paper, 14pt, twoside]{article}
\usepackage[14pt]{extsizes}
\usepackage[english,russian]{babel}
\usepackage[utf8]{inputenc}
\usepackage{amsfonts, amsmath, amssymb,amsthm, graphics,latexsym}
\usepackage{color}
\usepackage{wasysym}
\usepackage{misccorr}
\usepackage{mathrsfs}
\usepackage[active]{srcltx}
\usepackage{amsthm,fullpage}
\usepackage{fullpage}
\usepackage{multicol}
\usepackage[pdftex]{graphicx}
\usepackage[square, numbers]{natbib}
%\usepackage{fancyhrd}
%\usepackage{morefloats}
\usepackage{caption}
\usepackage{xcolor}
\usepackage{ifthen}
\usepackage{verbatim}
\usepackage{makeidx}
\usepackage{enumerate}
\usepackage{float}
\usepackage{graphicx,fancybox}
\usepackage{bm}
%\usepackage{indentfirst}
% \usepackage{commath}

\usepackage{hyperref} % For \url command

\usepackage{mathtools} % for def over equal sign command

\usepackage[makeroom,thicklines]{cancel} % For commands to cross-out math symbols. See http://tex.stackexchange.com/questions/75525/how-to-write-crossed-out-math-in-latex

\usepackage[left=3cm,right=2cm,
    top=3cm,bottom=3cm,bindingoffset=0cm]{geometry}
    
\usepackage{empheq} % for empheq environment
    
%\fontsize{14}{16pt}\selectfont \sloppy \allowdisplaybreaks
%%\renewcommand{\baselinestretch}{1.5}
%\sloppy


%%%%%%%%%%%%%%%%%% COLORS %%%%%%%%%%%%%%%%%%%%%%%%%%%%%
\renewcommand{\CancelColor}{\color{red}} % Color for canceling terms in expression

% Command for text highlighting
% Usegae:
%       \highlight{text}{color}
\newcommand{\highlight}[2]{
    \textbf{\color{#2}#1}}  % 

\newcommand{\todo}[1]{\textbf{\color{red}#1}}

%%%%%%%%%%%%%%%%%% MATH OPERATORS %%%%%%%%%%%%%%%%%%%%%%%%%%%%%
\DeclareMathOperator\dv{div} % divegence
\DeclareMathOperator\id{I}   % identity matrix
\DeclareMathOperator\tr{tr}  % trace operator
\DeclareMathOperator{\E}{E}  % Large strain tensor
\DeclareMathOperator{\const}{const} % const value 
\DeclareMathOperator{\rank}{rk}
%%%%%%%%%%%%%%%%%% BOLD SYMBOLS %%%%%%%%%%%%%%%%%%%%%%%%%%%%%
\newcommand{\bX}{\mathbf{X}}  % bold X
\newcommand{\bv}{\bm{v}}  % bold v
\newcommand{\bu}{\mathbf{u}}  % bold u
\newcommand{\bw}{\mathbf{w}}  % bold w
\newcommand{\bq}{\mathbf{q}}  % bold q
\newcommand{\bx}{\mathbf{x}}  % bold x
\newcommand{\be}{\mathbf{e}}  % bold e
\newcommand{\bs}{\mathbf{s}}  % bold s
\newcommand{\bn}{\mathbf{n}}  % bold n
\newcommand{\bp}{\mathbf{p}}  % bold p
\newcommand{\bff}{\mathbf{f}}  % bold f
\newcommand{\bc}{\mathbf{c}}  % bold c
\newcommand{\bbf}{\mathbf{f}}  % bold f
\newcommand{\bt}{\mathbf{t}} % bold T

\newcommand{\bD}{\mathbf{D}}  % bold D
\newcommand{\bP}{\mathbf{P}}  % bold P
\newcommand{\bcu}{\mathbf{U}} % bold U
\newcommand{\bct}{\mathbf{T}} % bold T

\newcommand{\bpsi}{\boldsymbol{\psi}} % bold psi
\newcommand{\bsigma}{\boldsymbol{\sigma}} % bold sigma
\newcommand{\btau}{\bm{\tau}} % bold tau
%%%%%%%%%%%%%%%%%% CALLIGRAPHIC SYMBOLS %%%%%%%%%%%%%%%%%%%%%%%%%%%%%
\newcommand{\epss}{{\cal E}}
\newcommand{\cL}{{\cal L}}
\newcommand{\cT}{{\cal T}}
\newcommand{\cQ}{{\cal Q}}

%%%%%%%%%%%%%%%%%% DIFFERENTIAL OPERATORS %%%%%%%%%%%%%%%%%%%%%%%%%%%%%
\newcommand{\pd}[2]{ % partial first derivative
   \dfrac{\partial #1}{\partial #2}
} 
\newcommand{\opd}[2]{ % ordinary first derivative
    \dfrac{\mathrm{d} #1}{\mathrm{d} #2}
}
\newcommand{\pddB}[2]{ % partial first derivative with function bordered by braces
    \dfrac{\partial }{\partial #2}\Bigl(#1\Bigr)
} 
\newcommand{\pdd}[3]{ % partial second mixed derivative
    \dfrac{\partial^2 #1}{\partial #2 \partial #3}
}

\newcommand{\pddfirst}[2]{ % partial second derivative with respect to one variable
    \frac{\partial^2 #1}{\partial {#2}^2}
}

%%%%%%%%%%%%%%%%%% NEW COMMANDS %%%%%%%%%%%%%%%%%%%%%%%%%%%%%
\newcommand\myeq{\stackrel{\mathclap{\footnotesize \mbox{def}}}{=}}

%%%%%%%%%%%%%%%%%% RENEW COMMANDS %%%%%%%%%%%%%%%%%%%%%%%%%%%%%
\renewcommand{\leq}{\leqslant} % Beautiful <=
\renewcommand{\geq}{\geqslant} % Beautiful >=

%%%%%%%%%%%%%%%%%% THEOREM STYLES %%%%%%%%%%%%%%%%%%%%%%%%%%%%%
\theoremstyle{definition}
\newtheorem{Lemma1}{Лемма}
\theoremstyle{definition}
\newtheorem{Problem}{Задача}
\newtheorem{Definition}{Определение}

\theoremstyle{remark}
\newtheorem*{Solution}{РЕШЕНИЕ}
\newtheorem*{Tip}{ПОДСКАЗКА}

%%%%%%%%%%%%%%%%%% PATHS TO FOLDERS WITH IMAGES %%%%%%%%%%%%%%%%%%%%%%%%%%%%%
\graphicspath{{img/}}

%%%%%%%%%%%%%%%%%% OTHER CUSTOMIZATIONS %%%%%%%%%%%%%%%%%%%%%%%%%%%%%
\everymath{\displaystyle} % Make all maths looks good

\DeclareMathOperator{\diver}{div}
\DeclareMathOperator\arctanh{arctanh}


%\textheight 240mm
%\textwidth 16cm
%\topmargin -15mm %было-15mm
%\multlinegap=0pt
\author{Калинин С.А., Кузнецов В.А., Стамлер К.В.}

%\title{Семинары по <<Введению в МСС>>}


